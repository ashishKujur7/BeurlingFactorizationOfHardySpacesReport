\chapter{Analytic and Harmonic Functions}
Let us begin by fixing the notation for the rest of the report. The \textit{open unit disc} will be denoted by $\D$ and is defined to the subset of the complex plane of those elements whose absolute value is less than $1$. That is,
\begin{equation*}
    \D = \left\{ z \in \C : \abs{z} < 1 \right\}.
\end{equation*}
The boundary of the open unit disc, called, the \textit{unit circle} is denotes by $\T$. That is, 
\begin{equation*}
    \T = \left\{ z \in \C : \abs{z} = 1 \right\}.
\end{equation*}

We will assume that the the unit circle has the normalised Lebesgue measure. The set of all complex measures denoted by $\calM \left( \T \right)$ is a Banach space with total variation norm. For a proof of this fact, we advise the reader to refer \cite{cohn2013measure}, \cite{axler2020measure} and \cite{rudin1987real}.

As usual, $\Omega$ will denote an open connected subset of $\C$. We will say that a function $U : \Omega \to \C$ is \textit{harmonic} on $\Omega$ if it satisfies the Laplace equation
\begin{equation*}
    \triangledown ^{2} U = \frac{\partial ^{2} U}{\partial x^{2}} + \frac{\partial ^{2} U}{\partial y^{2}} =0
\end{equation*}
for each point in $\Omega$.


Before we proceed, recall a basic fact from complex analysis which states that a real valued function on $\D$ is harmonic iff it is the real part an analytic function.

\section{The Poisson Kernel}
The famous Dirchlet problem asks the following:


Given $G \subset \R ^{2}$ and $f: \partial G \to \C$ and $f: \partial G \to \C$. Find a continuous function $u: \overline{G} \to \C$ such that $u\mid _{G}$  is harmonic and $u\mid_{\partial G} =f$.

To solve the above problem, we make the following definition:

\begin{definition}
    For each $0\le r < 1$, we define $P_r : \T \to \C$ by
    \begin{equation*}
	P_{r} \left( e^{it} \right) = \frac{1-r^{2}}{1+r^{2}-2r\cos t}
    \end{equation*}
    for each $e^{it} \in \T$.
    The family of functions $\left\{ P_{r} \right\}_{r\in [0,1)}$ is called the \textit{Poisson kernel} on the open unit disc $\D$.
    \label{def:Poisson-kernel}
\end{definition}

It can be shown that that for each $r<1$, we have that 
\begin{equation*}
    P_{r}\left( e^{it} \right) = \sum_{n=-\infty}^{\infty} r^{\abs{n}}e^{int}
\end{equation*}
holds for each $e^{it} \in \T$.

We will soon see that if $u \in \calC \left( \T \right)$, that is, continuous on $\T$, the so called "Poisson integral" of $u$ solves the Dirichlet problem. We postpone this work to the next section. However the following proposition may lead the reader to believe why this may be true:
\begin{proposition}
    Let $u: \overline{\D} \to \C$ be a harmonic function, this means, that $u$ is harmonic in an open subset containing $\overline{D}$. Then we have that
    \begin{equation}
	u\left( re^{i\theta} \right) = \frac{1}{2\pi} \int_{-\pi}^{\pi} u\left( e^{it} \right) P_{r} \left( e^{i\left( \theta -t \right)} \right) dt
    \end{equation}
    \label{prop:poisson-integral-formula-for-harmonic-function}
\end{proposition}

Albeit the proof of this proposition is not too hard, we ask the reader to seek a proof of this fact in \cite{hoffman2007banach}.

\section{Boundary Values}

The Poisson kernel is an "approximate identity". We just deal with the approximate identities on the circle:

\begin{definition}
    Let $\varphi_{i}$ be a subset of $L^{1} \left( \T \right)$ where $i$ varies over a directed set. We say that $\left\{ \varphi_{i} \right\}$ is an approximate identity on $\T$ if the following are satisfied:
    \begin{enumerate}
	\item for all $i$,
	    \begin{equation*}
		\frac{1}{2\pi} \int_{-\pi}^{\pi} \varphi_{i} \left( e^{it} \right) dt =1
	    \end{equation*}
	\item
	    \begin{equation*}
		C_{\varphi} = \sup_{i} \left( \frac{1}{2\pi} \int_{-\pi}^{\pi} \abs{\varphi_{i} \left( e^{it} \right)dt} \right) \le \infty
	    \end{equation*}
	\item for each fixed $\delta$, $0<\delta<\pi$, 
	    \begin{equation*}
		\lim_{i} \int_{\delta \le \abs{t} \le \pi} \abs{\varphi_{i} \left( e^{it} \right)} dt = 0.
	    \end{equation*}
    \end{enumerate}
    A family of approximate identity on $\T$ is called \textit{positive approximate identity} if all the function are positively valued, that is, 
    \begin{equation*}
	\varphi_{i}\left( e^{it} \right) \ge 0
    \end{equation*}
    for each $i$.
\end{definition}
As mentioned, the Poisson kernel is an example of approximate identity.The other well known approximate identity is Fejer's kernel on $\T$ which is essential in theory of convergence of Fourier series on $\T$. We refer the reader to \cite{katznelson2004introduction}.


The aforementioned theorems are esssential to the studying the interaction of $\D$ and the $\T$. We ask the reader to see \cite{mashreghi2009representation} for proofs for the sake of brevity. Although these theorems are done more generally, we really only care about the special case of the Poisson kernel.

\begin{theorem}[Weak* convergence of measures]
    Let $\left\{ \varphi _{i} \right\}_{i}$ be an approximate identity on $\T$ and let $\mu \in \calM \left( \T \right)$. Then for all $i$, $\varphi_{i} * \mu  \in L^{1} \left( \T \right)$ with
    \begin{equation*}
	\norm{\varphi _{i} * \mu}_{1} \le C_{\varphi} \lVert \mu \rVert
    \end{equation*}
    and
    \begin{equation*}
	\norm{\mu} \le \sup_{i} \norm{\varphi_{i} * \mu}_{1}\text{.}
    \end{equation*}
    Moreover, the measures $d\mu_{i} = \left( \varphi_{i} * \mu \right) \left( e^{it} \right) dt/2\pi$ converge to $d\mu \left( e^{it} \right)$ in the weak* topology, i.e.
    \begin{equation*}
	\lim_{i} \frac{1}{2\pi} \int_{-\pi}^{\pi} f\left( e^{it} \right) \left( \varphi_{i} * \mu \right) \left( e^{it} \right) dt = \int_{\T} \varphi\left( e^{it} \right) d\mu \left( e^{it} \right)
    \end{equation*}
    for all $f \in \calC \left( \T \right)$.
    \label{thm:weak-star-measures}
\end{theorem}
\begin{theorem}[Convergence in norm]
    Let $\left\{ \varphi _{i} \right\}_{i}$ be an approximate identity on $\T$ and let $f \in L^{p} \left( \T \right)$ with $p \in [1, \infty)$. Then for all $i$, $\varphi_{i} * f  \in L^{p} \left( \T \right)$ with
    \begin{equation*}
	\norm{\varphi _{i} * f}_{p} \le C_{\varphi} \lVert f \rVert _{p}
    \end{equation*}
    and
    \begin{equation*}
	\lim_{i} \norm{\varphi_{i} * f -f}_{p} = 0 \text{.}
	    \end{equation*} 
    \label{thm:convergence-in-Lp}
\end{theorem}


\begin{theorem}[Weak* convergence of bounded functions]
    Let $\left\{ \varphi _{i} \right\}_{i}$ be an approximate identity on $\T$ and let $f\in L^{\infty} \left( \T \right)$. Then for all $i$, $\varphi_{i} * \mu  \in \calC \left( \T \right)$ with
    \begin{equation*}
	\norm{\varphi _{i} * \mu}_{\infty} \le C_{\varphi} \lVert \mu \rVert _{\infty}
    \end{equation*}
    and
    \begin{equation*}
	\norm{f}_{+\infty} \le \sup_{i} \norm{\varphi_{i} * f}_{\infty}\text{.}
    \end{equation*}
    Moreover, $\varphi_{i} * f$ converge to $f$ in the weak* topology, i.e.
    \begin{equation*}
    \lim_{i} \int_{-\pi}^{\pi} g\left( e^{it} \right) \left( \varphi_{i} * f \right) \left( e^{it} \right) dt = \int_{\T} g\left( e^{it} \right) f \left( e^{it} \right) dt 
    \end{equation*}
    for all $g \in L^{1} \left( \T \right)$.
    \label{thm:weak-star-infinity}
\end{theorem}

\subsection{The specific case of Poisson Integral}

\begin{definition}[Poisson integral of some function or measure]
    Let $\tilde{f} : \D \to \C$ be a harmonic function. Then $\tilde{f}$ is said to be the \textit{Poisson integral} of the function $f : \T \to \C$ if
    \begin{equation*}
	\tilde{f} (re^{i\theta}) = \frac{1}{2\pi} \int_{T} f\left( e^{it} \right) P_{r} \left( e^{i\left( \theta-t \right)} \right) dt
    \end{equation*}
    In such a case, we will denote the function $\tilde{f}$ by $P[f]$.
    Similarly, $f$ is said to be the \textit{Poisson integral} of a complex measure $\mu$ on $T$ if
\begin{equation*}
    \tilde{f} (re^{i\theta}) = \frac{1}{2\pi} \int_{T} P_{r} \left( e^{i\left( \theta-t \right)} \right) d\mu\left( e^{it} \right)
    \end{equation*}In such a case, we will denote the function $\tilde{f}$ by $P[\mu]$.
    \label{def:Poisson-Integral-Of-Some-Function-Or-Measure}
\end{definition}

This follows immediately by noticing that the family of Poisson kernel is indeed an approximate identity. 
\begin{theorem}
    Let $f : \D \to \C$ be a harmonic function. Define for each $r\in [0,1)$, the function $f_{r} : \T \to \C$ by
    \begin{equation*}
	f_{r} \left( e^{i\theta} \right) = f\left( re^{i\theta} \right)
    \end{equation*}
    The following statements holds:
    \begin{enumerate}
	\item If $1 < p \le \infty$ then $f=P[g]$ for some $g \in L^{p} [g]$ iff for each $r > 0$, $\norm{f_{r}}_{p} < +\infty$ .
	\item If p=1 then $f=P[g]$ for some $g \in L^{p} [g]$ iff $f_{r}$ converge in the $L^{1}$ norm.
	\item $f=P[\mu]$ for some $\mu \in \calM (\T)$ iff for each $r > 0$, $\norm{f_{r}}_{1} < +\infty$ 
    \end{enumerate}
    \label{thm:convergence-Poisson}
\end{theorem}
\section{Fatou's Theorem}

The next theorem requires the reader to know what derivative of measures are. The reader can get review of these concepts in \cite{rudin1987real} and \cite{cohn2013measure}.

\begin{theorem}[Fatou (1905)]
    Let $\mu$ be a complex measure on the unit circle $\T$, and let $f: \D \to \C$ be the harmonic function defined by
    \begin{align*}
	f(re^{i\theta}) = \frac{1}{2\pi} \int_{\T} P_{r} \left( e^{i\left( \theta-t \right)} \right) d\mu \left( e^{it} \right)
    \end{align*}

    Let $e^{i\theta_{0}}$ be any point where $\mu$ is differentiable with respect to the normalised Lebesgue measure. Then
    \begin{equation*}
	\lim_{r\to 1} f\left( re^{i\theta_{0}} \right) = \left( \frac{d\mu}{d\theta} \right) \left( e^{i\theta _{0}} \right) = \mu ' \left( e^{i\theta _{0}} \right)
    \end{equation*}
    In fact, $f(re^{i\theta}) \to \mu ' \left( e^{i\theta_{0}} \right)$ as $re^{i\theta}$ approaches $e^{i\theta_{0}}$ along any path in the open disc within the region of the form $\abs{\theta - \theta_{0}} \le c \left( 1-r \right)$ for some $c> 0$. 
    \label{thm:Fatou-1906}
\end{theorem}

A proof of the above theorem is quite technical and can be found in \cite{koosis1998introduction}, \cite{hoffman2007banach} and \cite{axler2013harmonic}.

\begin{corollary}
    Let $\mu$ be a complex measure on $\T$. Then $P[\mu]$ has nontangential limits equal everywhere to the Radon Nikodym derivative of $\mu$ with respect to the normalised Lebesgue measure.
\end{corollary}
\begin{proof}
    Let $\mu$ be a complex measure on the unit circle. Then by LebesgueDecomposition Theorem, we have that

    \begin{equation*}
	d\mu = \frac{1}{f} dt + d\mu_{s}
    \end{equation*}
    where $d\mu_{s}$ is a singular measure. 
    Since $\mu$ is a complex measure, hence, its real and imaginary parts are finite measures, we have that $\mu$ is differentiable almost everywhere and we have that
    \begin{equation*}
	\frac{d\mu}{dt} = \frac{1}{2\pi}f
    \end{equation*}
    almost everywhere.

    Applying Fatou's theorem \ref{thm:Fatou-1906}, this follows immediately.
\end{proof}

\begin{corollary}
    Let $f : \T \to \C$ be $L^{1}$. Then $P[f]$ has nontangential limits at almost everywhere and these limits equal to $f$ almost everywhere.
    \label{cor:L1-implies-Poisson-limits}
\end{corollary}
\begin{proof}
    This corollary follows immediately by considering the measure on the circle given by
    \begin{equation*}
	d\mu = f \frac{dt}{2\pi}
    \end{equation*}
    Applying the previous corollary, we are done.
\end{proof}

The next corollary tells us that if the $p$ norm of $f_{r}$ is uniform bounded as $r$ varies, the $f$ is a Poisson integral of a $L^{p}$ function on the circle or a complex measure on the circle depending upon what $p$ is.
\begin{corollary}
    Let $f: \D \to \C$ be a harmonic function and $1\le p <\infty$. Suppose that for all $0\le r < 1$, we have that
    \begin{equation*}
	\norm{f_{r}}_{p} < M< +\infty
    \end{equation*}
    Then for almost every $\theta$ the radial limits 
    \begin{equation*}
	\tilde {f} (e^{i\theta} ) = \lim_{r\to 1} f\left( re^{i\theta} \right)
    \end{equation*}
    exist and define a function $\tilde f$ in $L^{p} \left( \T \right)$. The following also holds:
    \begin{enumerate}
	\item If $p>1$ then $f=P[\tilde{f}]$.
	\item If $p=1$ then $f=P[\mu]$ for some complex measure $\mu$ whose absolutely continuous part is $fd\theta$.
	\item IF $f$ is bounded then the boundary values exist almost everywhere and define a bounded measurable function $\tilde{f}$ on $\T$ such that $f=P[\tilde{f}]$.
    \end{enumerate}
    \label{cor:imp-Fatou}
\end{corollary}



\begin{proof}
    Suppose that for each $r\in [0,1)$, we have $\norm{f_{r}}_{p} < +\infty$. We need to prove that for almost every $\theta$, $\lim_{r\to 1} f\left( re^{i\theta} \right)$ exists. Then by Theorem \ref{thm:convergence-Poisson}, we have that $f=P[g]$ for some $g\in L^{p} \left( \T \right)$. Since $L^{p} \left( \T \right) \subset L^{1} \left( \T \right)$, we can use the previous corollary. By the previous corollary, we have that $P[g]$ has nontangential limits almost everywhere, we have that
    \begin{equation}
	\tilde{f} \left( e^{i\theta} \right) = \lim_{r\to 1} f(re^{i\theta}) = \lim_{r\to 1} P[g] \left( re^{i\theta} \right)
	\label{eqn:radial-limits}
    \end{equation}
    exists almost everywhere.

    Now we proceed to prove part $(1)$. Also by Theorem \ref{thm:convergence-Poisson}, we have that $f=P[g]$ for some $g\in L^{p} \left( \T \right)$. Hence, we have that by Equation \ref{eqn:radial-limits} that $\tilde{f} (e^{i\theta}) =  \lim_{r\to 1} P[g] \left( re^{i\theta} \right)$ holds at almost every $\theta$.

    Also, by the previous corollary, $\lim_{r\to 1} P[g] \left( re^{i\theta} \right) = g(e^{i\theta})$ for almost every $\theta$. Hence, we have that $\tilde{f} = g$.

\end{proof}

\begin{corollary}
    Let $f: \D \to \R_{\ge 0}$ be a harmonic function. Then $f$ has nontangential limits at almost every point of $\T$.
    \label{cor:nontangential-limits}
\end{corollary}
\begin{proof}
    The proof just relies on \ref{prop:poisson-integral-formula-for-harmonic-function}.
\end{proof}

Let $h \left( \D \right)$ denote the set of all harmonic functions on $\D$. Let $p\in [1,\infty]$. Define
\begin{equation*}
    h^{p} \left( \D \right) = \left\{ f\in h \left( \D \right) \, \mid \, \left\{ f_{r} \right\}_{0\le r < 1} \text{ is uniformly bounded in } L^{p} \text{ norm }\right\}
\end{equation*}
We define a norm on $h^{p} \left( \D \right)$ by
\begin{equation*}
    \norm{f}_{h^{p} \left( \D \right)} = \sup_{0\le r < 1} \norm{f_{r}}_{L^p \left( \D \right)} =\begin{cases} \sup_{0\le r < 1} \left( \frac{1}{2\pi} \int_{-\pi}^{\pi} \abs{f\left( re^{i\theta}\right)}^{p}d\theta   \right)^{\frac{1}{p}} & \text{if } p \in [1, \infty) \\
	\sup_{0\le r < 1} \norm{f(re^{i\theta})}_{L^{\infty} \left( \D \right)} & \text{ if } p=\infty
    \end{cases}
\end{equation*}

It is easy to see why $\norm{f} < +\infty$ for any $f\in h^{p} \left( D \right)$. So we now proceed to show that $h^{p} \left( D \right)$ is a Banach space with this norm. First we show that it is indeed a normed linear space.

Clearly, $h \left( \D \right)$ is a vector space. To show that $h^{p} \left( \D \right)$ is a vector space, it suffices to check that $h^{p} \left( \D \right)$ is a subspace.

Let $f,g \in h^{p} \left( \D \right)$ and let $\alpha \in \C$. Then for any $r\in [0,1)$, we have that 
\begin{align*}
    \norm{(f+\alpha g)_{r}}_{p} &= \norm{f_{r} + \alpha g_{r}} \\
    &= \norm{f_{r}}_{p} + \alpha \norm{g_{r}}_{p}
\end{align*}
Take note of the use of Holder's inequality. After this is done, since $\left\{ f_{r} \right\}_{r\in [0,1)}$ and $\left\{ g_{r} \right\}_{r\in [0,1)}$ is uniformly bounded, we have that $\left\{ f+ \alpha g \right\}_{r\in [0,1)}$ is uniformly bounded in $L^{p}$ norm.

Now, we need to show that it is a normed linear space but this follows almost immediately.

To show that it is a Banach space, we show that

\begin{theorem}
    Let $p\in [1, \infty]$. If $u\in L^{p} \left( \T \right)$ then $f=P * u \in h^{p} \left( \D \right)$ and $\norm{f}_{p} =\norm{u}_{p}$. If $\mu \in \calM \left( \T \right)$ then $f= P * \mu \in h^{1} \left( \D \right)$ and $\norm{f}_{1} = \norm{\mu}$.
    \label{thm:lp-and-hp}
\end{theorem}
\begin{proof}
    We consider the case $p \in [1, \infty )$. The other cases can be dealt similarly. Consider the map 
    \begin{equation*}
	u \stackrel{T}{\mapsto} U
    \end{equation*}
    where $U\left( re^{i\theta} \right) = \frac{1}{2\pi} \int_{-\pi}^{\pi} P_{r}\left( e^{i(\theta-t)} \right) u \left( e^{it} \right) dt$. By Theorem \ref{thm:convergence-in-Lp}, we have that $\norm{U} = \norm{u}_{p} < +\infty$. Hence $U \in h^{p} \left( \D \right)$.

    Linearity is obvious. We need to check injectivity and surjectivity.

    To check injectivity, let $u \in L^{p} \left( \T \right)$ and suppose that $T(u)=P[u]=0$. Now $\lim_{r\to 1} P[u] \left( re^{i\theta} \right) = u$ for almost $\theta$ by Corollary \ref{cor:L1-implies-Poisson-limits} and hence $u=0$ almost everywhere.

    Surjectivity is clear from Theorem \ref{thm:convergence-Poisson}.
    \end{proof}

%%%%%%%%%%%%%%%%%%%%%%%%%%%%%%%%%%%%%%%%%%%%%%%%%%%%%%%%%%%%%%%%%%%%%%%%%%%%%%

\section{Hardy Spaces -- \texorpdfstring{$H^p$}{\text{Hp}} spaces}

Let us denote the set of all analytic functions on $\D$ by $H \left( \D \right)$. Hence, $H \left( \D \right) \subset h \left( \D \right)$. For $p\in (0,\infty]$, we consider the \textit{Hardy classes} of analytic functions on the unit disc

\begin{equation*}
    H^{p} \left( \D \right) = \left\{ F \in H \left( \D \right) \, \mid \, \norm{F} _{p} < \infty \right\}
\end{equation*}

Clearly, 
\begin{align*}
    H^{p} \left( \D \right) \subset h ^{p} \left( \D \right)
\end{align*}

We will see that $H ^{p} \left( \D \right)$, $1\le p \le +\infty$, is also a Banach spaces isomorphic to a closed subspace of $L^{p} \left( \T \right)$ denotes by $H^{p} \left( \T \right)$.

To prove that $H^{p} \left( \D \right)$ is a closed subspace of $h^{p} \left( \D \right)$, we are going to identify $H^{p} \left( \D \right)$ with the closed subspace
\begin{align*}
    \left\{ u\in L^{p} \left( \T \right) : \int_{-\pi}^{\pi} u\left( e^{it} \right) e^{ikt} = 0 \textit{ for all } k \in \N \right\}
\end{align*}

Let $\left\{ u_{n} \right\}$ be a sequence of functions in the above subspace; suppose that $\left\{ u_{n} \right\}$ converge to $u\in L^{p} \left( \T \right)$. Now, let $k\in \N$ be arbitrary. Since $\left\{ u_{n} \right\}$ converge to $u$ in $p$-norm, we have that $\left\{ u_{n} \right\}$ converge to $u$ in $1$-norm. Hence we have the following:
\begin{align*}
    \abs{\int_{-\pi}^{\pi} u_{n} \left( e^{it} \right) e^{ikt} dt - \int_{-\pi}^{\pi} u \left( e^{it} \right) e^{ikt} dt } &\le \int_{-\pi}^{\pi} \abs{u_{n} \left( e^{it} \right) - u\left( e^{it} \right)} dt
\end{align*}
From the above inequality, it is evident that $u$ is in the subspace mentioned above.

For the above reasons, we will not distinguish between the $H^{p} \left( \D \right)$ and $H^{p}\left( \T \right)$ and simply write $H^{p}\left( \D \right)$. The case $p=1$ needs a special mention. We have not dealt with it yet. We saw that if $f \in H^{1}\left( \D \right)$ then we have that 
\begin{equation*}
    f = P[\mu]
\end{equation*}
for some complex measure $\mu$. We do not recover a $f\in L^{1} \left( \T \right)$ like the other case. In this case, we know that this measure $\mu$ satisfies the property that
\begin{equation*}
    \int_{-\pi}^{\pi} e^{int} dt = 0
\end{equation*}
for each $n=1,2,3,\ldots$. In other words, we can say the negative Fourier coefficient of $\mu$ is zero. We will call a such a complex measure to be \textit{analytic}.

In the next chapter, we will see that a theorem due to F and M Riesz says that complex analytic measure is absolutely continuous with respect to the normalised Lebesgue measure. That is, there is a $\tilde{f}\in L^{1} \left( \T \right)$ such that 
\begin{equation*}
    d\mu = \tilde{f} \frac{dt}{2\pi}.
\end{equation*}

In the proof of Theorem \ref{thm:lp-and-hp}, we can send $f$ to $\tilde{f}$ to establish a bijection and we will be done.

\subsubsection{Series Representation of Harmonic Functions}

The proof of the aforementioned theorem is used later and not necessary for our discussion. We refer the reader to \cite{mashreghi2009representation} for a proof.

\begin{theorem}
    Let $U$ be a harmonic on the disc $D_{R} = \left\{ |z| < R \right\}$. Then, for each $n\in \Z$, the quantity
    \begin{equation}
	a_{n} = \frac{\rho^{-\abs{n}}}{2\pi} \int_{-\pi}^{\pi} U\left( \rho e^{it} \right) e^{-int} dt \qquad \left( 0 < \rho < R \right)
	\label{eqn:srhf-1}
    \end{equation}
    is independent of $\rho$ and we have
    \begin{equation}
	U\left( re^{i\theta} \right) = \sum_{n=-\infty}^{\infty} a_{n} r^{\abs{n}} e^{in\theta} \qquad \left( re^{i\theta} \in \D \right)
	\label{eqn:srhf-2}
    \end{equation}
    The function
    \begin{equation}
	V\left( re^{i\theta} \right) = \sum_{n=-\infty}^{\infty} -i \text{sgn } (n) a_{n} r^{\abs{n}} e^{in\theta} \qquad \left( re^{i\theta} \in \D \right)
	\label{eqn:srhf-3}
    \end{equation}
    is the unique harmonic conjugate of $U$ such that $V\left( 0 \right)  = 0$. The series in \ref{eqn:srhf-2} and \ref{eqn:srhf-3} are absolutely and uniformly convergent on compact subsets of $D_{R}$
    \label{thm:srhf}
\end{theorem}
